\documentclass[review]{elsarticle}
\usepackage[utf8x]{inputenc}
\usepackage[vietnam]{babel}


\usepackage{lineno,hyperref}
\modulolinenumbers[5]

\journal{Journal of \LaTeX\ Templates}

%%%%%%%%%%%%%%%%%%%%%%%
%% Elsevier bibliography styles
%%%%%%%%%%%%%%%%%%%%%%%
%% To change the style, put a % in front of the second line of the current style and
%% remove the % from the second line of the style you would like to use.
%%%%%%%%%%%%%%%%%%%%%%%

%% Numbered
%\bibliographystyle{model1-num-names}

%% Numbered without titles
%\bibliographystyle{model1a-num-names}

%% Harvard
%\bibliographystyle{model2-names.bst}\biboptions{authoryear}

%% Vancouver numbered
%\usepackage{numcompress}\bibliographystyle{model3-num-names}

%% Vancouver name/year
%\usepackage{numcompress}\bibliographystyle{model4-names}\biboptions{authoryear}

%% APA style
%\bibliographystyle{model5-names}\biboptions{authoryear}

%% AMA style
%\usepackage{numcompress}\bibliographystyle{model6-num-names}

%% `Elsevier LaTeX' style
\bibliographystyle{elsarticle-num}
%%%%%%%%%%%%%%%%%%%%%%%

\begin{document}

\begin{frontmatter}

\title{Tài liệu kĩ thuật về chương trình phân lớp}


%% Group authors per affiliation:
\author{Mai Tiến Dũng}
\address{Radarweg 29, Amsterdam}

\begin{abstract}
Tài liệu giới thiệu về các thư viện, toolkit, các vấn đề nảy sinh và cách giải quyết khi cài đặt chương trình phân lớp ảnh trong trường hợp số lượng lớp và số lượng ảnh lớn.
\end{abstract}

\begin{keyword}
phân lớp ảnh
\end{keyword}

\end{frontmatter}

\linenumbers

\section{Giới thiệu bài toán phân lớp}
Cho trước một tập gồm $N$ ảnh đã được phân vào một trong $C$ lớp.
Bài toán: xây dựng mô hình phân lớp sao cho có thể dự đoán một ảnh mới thuộc vào lớp nào trong $C$ lớp.

Tiêu chí đánh giá: độ chính xác và thời gian phân lớp.

\section{Biểu diễn ảnh}

Mỗi ảnh được biểu diễn bằng một vector đặc trưng. Trong thực nghiệm này, chúng tôi sử dụng mô hình bag-of-word (bag-of-feature)



Sử dụng toolkit The encoding methods evaluation toolkit 

- Thư viện nào cần dùng (ví dụ vlfeat version xx, LLC, etc)

- Thư viện code đã viết (đường dẫn trên github, mô tả cụ thể các code chính)

- Các datasets, với các số liệu chi tiết

- Các vấn đề nảy sinh và cách giải quyết

- Các kết quả liên quan đến thời gian xử lí, độ chính xác, so sánh với các pp khác.

- Càng chi tiết càng tốt, có hình minh họa, bảng biểu.

- Có thể sử dụng lại các log của các task trước đó.

- Báo cáo viết bằng latex.
\section{Lưu ý}
-Date: Nov-27-2014:
\begin{itemize}
\item Các classifier đã train từ các feature có được bằng codebook A thì bắt buộc các ảnh test cũng phải dùng code book A, nếu dùng codebook khác sẽ có kết quả rất khác.
\item cùng (codebook, kdtree và feat) thì hàm pooler tính ra feature là như nhau
\item cùng (codebook, kdtree)  sau đó tính (feat, pooler) tính ra feature là như nhau
\item cùng codebook, nhưng kdtree khác nhau sẽ ra feature khác nhau.
\item Cùng 1 code book, nhưng hàm kdtree = vl_kdtreebuild(codebook) sẽ tạo ra kdtree khác nhau trong các lần gọi hàm khác nhau.
\item Nếu không dùng kdtree thì tính feature giữa các lần khác nhau cho cùng 1 ảnh là giống nhau (mất thời gian tính toán feature).

\end{itemize}

 

\section{The Elsevier article class}

\paragraph{Installation} If the document class \emph{elsarticle} is not available on your computer, you can download and install the system package \emph{texlive-publishers} (Linux) or install the \LaTeX\ package \emph{elsarticle} using the package manager of your \TeX\ installation, which is typically \TeX\ Live or Mik\TeX.

\paragraph{Usage} Once the package is properly installed, you can use the document class \emph{elsarticle} to create a manuscript. Please make sure that your manuscript follows the guidelines in the Guide for Authors of the relevant journal. It is not necessary to typeset your manuscript in exactly the same way as an article, unless you are submitting to a camera-ready copy (CRC) journal.

\paragraph{Functionality} The Elsevier article class is based on the standard article class and supports almost all of the functionality of that class. In addition, it features commands and options to format the
\begin{itemize}
\item document style
\item baselineskip
\item front matter
\item keywords and MSC codes
\item theorems, definitions and proofs
\item lables of enumerations
\item citation style and labeling.
\end{itemize}

\section{Front matter}

The author names and affiliations could be formatted in two ways:
\begin{enumerate}[(1)]
\item Group the authors per affiliation.
\item Use footnotes to indicate the affiliations.
\end{enumerate}
See the front matter of this document for examples. You are recommended to conform your choice to the journal you are submitting to.

\section{Bibliography styles}

There are various bibliography styles available. You can select the style of your choice in the preamble of this document. These styles are Elsevier styles based on standard styles like Harvard and Vancouver. Please use Bib\TeX\ to generate your bibliography and include DOIs whenever available.

Here are two sample references: \cite{Feynman1963118,Dirac1953888}.

\section*{References}

\bibliography{mybibfile}

\end{document}